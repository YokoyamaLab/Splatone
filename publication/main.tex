% !TeX program = pdflatex
% !TeX encoding = UTF-8

%==============================
% Splatone Preprint (Skeleton)
%==============================

\documentclass[11pt,a4paper]{article}

%--------- Packages ----------
\usepackage[T1]{fontenc}
\usepackage[utf8]{inputenc}
\usepackage{lmodern}
\usepackage{amsmath,amssymb,amsthm}
\usepackage{graphicx}
\usepackage{authblk}
\usepackage{url}
\usepackage{hyperref}
\usepackage{geometry}
\geometry{margin=25mm}

%--------- Meta data ----------
\title{Splatone: A Visual Analytics Tool for Geotagged Image Collections}

% TODO: 著者情報を埋めてください
\author[1]{First Author}
\author[1]{Second Author}
\affil[1]{Affiliation, Address\\
  \texttt{email@example.com}}

\date{\today}

%--------- Document ----------
\begin{document}

\maketitle

\begin{abstract}
  % このツール Splatone の概要を1段落で記述してください。
  This paper presents Splatone, a tool for visual exploration
  of large geotagged image collections using color-based spatial
  visualizations.
\end{abstract}

\section{Introduction}
% ・問題設定: 位置情報付き画像の可視化・解析の課題
% ・Splatone の目的と貢献
% ・本論文の構成

\section{Related Work}
% 位置情報付き画像の可視化, カラーパレット生成, マップ・地理可視化などを整理

\section{System Overview}
\subsection{Data Model}
% 画像, メタデータ (位置, タグなど) の扱い

\subsection{Architecture}
% Crawler, Plugins, Visualization Pipeline など Splatone の構成概要

\section{Color-based Visualization}
\subsection{Palette Generation}
% `lib/paletteGenerator.js` でのパレット生成アルゴリズム概要

\subsection{Visual Encodings}
% voronoi, pie-charts, marker-cluster などのビジュアライザ概要

\section{Implementation}
% Node.js ベースの実装, プラグイン機構, Web UI など

\section{Use Cases}
% 例: Flickr 画像を用いた都市・景観の色分布解析など

\section{Discussion}
% 利点・限界・今後の展望

\section{Conclusion}
% 本ツールのまとめと今後の課題

\section*{Acknowledgements}
% 謝辞があれば記載

\bibliographystyle{plain}
\bibliography{references}

\end{document}
